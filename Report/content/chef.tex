\section{Chef}
	Short intro to puppet and key design principles
	Found on \url{https://docs.chef.io/chef_overview.html}
	\begin{itemize}
		\item Components
			\begin{itemize}
				\item Chef DK (Chef Development Kit)
					\begin{itemize}
						\item Computers running Chef DK are called Workstations
						\item Creation of cookbooks
						\item Test of cookbooks with Test Kitchen
							\begin{itemize}
								\item Describe Test Kitchen here
							\end{itemize}
						\item Components of workstations
							\begin{itemize}
								\item Knife
									\begin{itemize}
										\item Interface between local chef-repo and Chef server
									\end{itemize}
								\item The chef-repo
									\begin{itemize}
										\item Cookbook storage
										\item "The chef-repo should be synchronized with a version control system (such as git), and then managed as if it were source code"\\ {\tiny \url{https://docs.chef.io/workstation.html#configure-ruby-environment}}
									\end{itemize}
								\item knife.rb
									\begin{itemize}
										\item File to specify configuration details for knife
									\end{itemize}
							\end{itemize}
					\end{itemize}
				\item Chef Server
					\begin{itemize}
						\item Hub for configuration data (cookbooks)
						\item Pull configuration: Nodes pull cookbooks from server
						\item Features
							\begin{itemize}
								\item Search any type of data that is indexed by the Chef server (features wildcards, etc.)
								\item Management of
								\begin{itemize}
									\item Nodes
									\item Cookbooks and recipes
									\item Roles
									\item Stores of JSON data (data bags), including encrypted data
									\item Environments
									\item User accounts and user data
								\end{itemize}
								\item data bag
									\begin{itemize}
										\item Global variable that is stored as JSON data
										\item Accessible from Chef server
										\item Indexed for searching
									\end{itemize}
								\item Policy
									 \begin{itemize}
									 	\item Role
										 	\begin{itemize}
										 		\item Defines patterns and processes that exists across nodes
										 		\item Chef client merges attributes and run-lists with assigned roles
										 	\end{itemize}
									 \end{itemize}
								\item Environment
									\begin{itemize}
										\item Maps the real-life workflow to the configuration items of Chef
										\item Can be associated with one or more cookbook versions
									\end{itemize}
								\item Run-list
									\begin{itemize}
										\item Ordered list of roles and/or recipes
										\item Items run in the order defined in the run-list
										\item Can be node-specific
										\item Stored as part of the node object on the Chef server
										\item Maintenance with knife or Chef Automate
									\end{itemize}
							\end{itemize}
					\end{itemize}
				\item Chef client
					\begin{itemize}
						\item Must be installed on each node
						\item Performs
							\begin{itemize}
								\item Registration and authentication of the node with the chef server
								\item Building the node object
								\item Synchronization of cookbooks
								\item Compilation of the resource collection
								\item Configuration of the node
								\item Exception and notification handling
							\end{itemize}
					\end{itemize}
				\item Ohai
					\begin{itemize}
						\item Collects system configuration data for use within cookbooks
						\item Includes many built-in plugins to detect state
						\item Attributes contain: OS, Network, Memory, Disk, CPU, Kernel, host names, virtualization, etc.
					\end{itemize}
				\item Chef Supermarket
				\begin{itemize}
					\item Sharing and management of community cookbooks
				\end{itemize}
			\end{itemize}
		\item Cookbooks contain
			\begin{itemize}
				\item attributes
				\item cookbook\_file
				\item libraries: Ruby code can be included in a cookbook
				\item metadata: Stored in \textit{metadata.rb}. Helps the server deploy the cookbooks to the nodes correctly
				\item recipes
					\begin{itemize}
						\item Authored in Ruby
						\item Collection of ressources
						\item Must define everything that is needed to configure the node
					\end{itemize}
				\item ressources
					\begin{itemize}
						\item Describes the desired state for a configuration item
						\item Describes the steps to achieve the desired state
						\item Contains ressource type
						\item Grouped into recipes
					\end{itemize}
				\item templates
					\begin{itemize}
						\item Used to dynamically generate static text files
						\item May contain Ruby
						\item Intended to manage configuration files
					\end{itemize}
				\item tests
			\end{itemize}
	\end{itemize}