\subsection{Puppet}

	\begin{itemize}
		\item Short introduction to history (maybe?)
		\item Idea: Balance between detail-hiding and required customization
	\end{itemize}

	\begin{itemize}
		\item Consists of Puppet server and Puppet agents
		\item Server (\url{https://puppet.com/docs/puppet/4.9/architecture.html})
		\begin{itemize}
			\item One or more servers can run the Puppet master application
			\item 
		\end{itemize}
		\item Uses a master-slave architecture
		\item Uses pull config
		\item Ressource management:
		\begin{itemize}
			\item Manifests describe the node configuration
			\item Groups of ressources can be organized into classes $\Rightarrow$ i.e. config for entire application can be grouped
			\item Modules combine manifests and data to improve code organization
		\end{itemize}
		\item Server node connection via SSL works as follows:
		\begin{enumerate}
			\item Node sends normalized data, called facts, to the Puppet master and requests a catalog
			\item Communication between Server and Client via HTTPS and client-verification (SSL certificate)
			\item Server uses this data to compile a catalog, that specifies how the node should be configured
			\item After applying the catalog, the agent submits a report to the Puppet master: The node reports back the successful config to the master (Visible on the Puppet Dashboard)
		\end{enumerate}
		\item Puppet can run as stand-alone architecture
		\item Puppet language
		\begin{itemize}
			\item \textbf{Declarative} language
			\item \textbf{Resource} can describe a single file or packet
			\item Groups of resources can be organized as \textbf{Classes}
			\item Files
			\begin{itemize}
				\item Files are called \textbf{Manifests}
				\item Are named with \inlinedef{.pp}
				\item Must use UTF-8
				\item File example
				\begin{lstlisting}[language=Ruby]
					case $operatingsystem {
					centos, redhat: { $service_name = 'ntpd' }
					debian, ubuntu: { $service_name = 'ntp' }
					}
					
					package { 'ntp':
					ensure => installed,
					}
					
					service { 'ntp':
					name      => $service_name,
					ensure    => running,
					enable    => true,
					subscribe => File['ntp.conf'],
					}
					
					file { 'ntp.conf':
					path    => '/etc/ntp.conf',
					ensure  => file,
					require => Package['ntp'],
					source  => "puppet:///modules/ntp/ntp.conf",
					# This source file would be located on the Puppet master at
					# /etc/puppetlabs/code/modules/ntp/files/ntp.conf
					}
				\end{lstlisting}
			\end{itemize}
		\end{itemize}
	\end{itemize}
q