% WRITTEN BY: Both
\section{Conclusions}

While originally providing different approaches to CM, Chef and Puppet have converged in terms of feature set over time. They have implemented a client-server architecture based approach and offer a variety of tools to abstract infrastructure configuration from client specific details to a generalized, abstract approach. High efficiency is provided though Forge and Supermarket, that enable sharing components and encourage reusing existing configurations. While offering a free, open source package, commercial support is also available to fit the customer's needs in terms of long term support, troubleshooting and training opportunities.

\todo{comparison table? Conways law?}

In conclusion, both tools provide a reliable platform for managing critical infrastructure. In the end, the decision of or against a certain tool might not only be technical but depend on the environment for which it is intended to work in. If so available, the existing hard- and software must be taken into account in terms of expenditure required in the conversion process. If a CM tool is already in use, how difficult will the adoption process be and how similar is the form in which the infrastructure is described as code?

Usually CM is not limited to performing the actual configuration itself but involve the integration of external tools in order to perform i.e. testing configuration code or provide monitoring capabilities of the managed nodes. If so necessary, availability and compatibility with such tools must be ensured during the assessment of suitable software solutions.

Apart from the technical architecture itself, it must also be considered how well the system administrators are already familiar with the desired tool and its properties. If no previous experience exists with a tool that requires a steep learning curve, the initial progress may prove to be slow and the chance of configuration errors must be taken into consideration.