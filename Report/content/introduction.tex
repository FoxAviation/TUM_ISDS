\section{Introduction}

	The system administrators task of configuring, monitoring and maintaining modern computer systems requires extensive knowledge of setup and troubleshooting procedures as well as in-depth technical knowledge on the systems components (databases, networks, applications, load balancers) and their respective interactions \cite{Barrett:2004:FSC:1031607.1031672}. While virtualization decouples compute resources from the physical hosts the system operates, administering complex systems remains a daunting task. To tackle the challenge of administrating heterogeneous IT infrastructures with systems distributed across multiple servers and networks, automation tooling has been developed to alleviate issues associated with managing such infrastructures \cite{Hintsch2016ARO}. \\ This section introduces the concept of configuration management tools and typical problems solved through such software.

\subsection{Configuration Management (CM)}

Configuration management tools act as an operating system abstraction layer (OSAL), enabling the management of computational resources using automated processes defined in code, a paradigm often referred to as \textit{Infrastructure as Code}. Instead of provisioning and operating computer systems through hardware and manual command-line interaction, specialized tools and definition formats were developed to make knowledge about the setup of computing environments explicit in a machine-readable format.

\subsection{Infrastructure as Code}

Even before the emergence of CM tools, common tasks to run and maintain infrastructures consisting of networks, computers, storage servers, firewalls and monitoring systems were automated, usually through a plethora of scripts implementing the desired functionality specific for the infrastructure at hand. Often times, these environments would have to undergo significant efforts to rebuild their infrastructure from scratch in case of a disaster since their is no explicit description of the environments configuration available \cite{Hüttermann2012}.

Large-scale infrastructures constantly evolve and manual configuration is not sustainable in these settings. This is why many concepts from software development such as reusability and composability were adopted to create tools which make infrastructure configuration similar to configuring software \cite{kanies2006puppet}. This enables the use of version control systems (VCS) for managing infrastructure in a collaborative manner like they are used in many software development projects.

\subsection{Reasons for Configuration Management}

The use of a configuration management system allows to manage the evolution of an IT infrastructure similiar to the evolution of a codebase. Many of the reasons for using version control systems to manage code \cite{Dart:1991:CCM:111062.111063}\cite{1983ansi} carry over to the use of modern software configuration tools.

\begin{description}

\item[Identification] \hfill \\ 
An identification scheme uniquely identifies every infrastructure component, their type and their configuration. 

\item[Control] \hfill \\
Changes to the infrastructure are applied through an automated process
ensuring the consistent and correct application of changes.

\item[Status Accounting]  \hfill \\
Current system status and changes can be traced and used for troubleshooting of components or infrastructure improvement planning.

\item[Audit and Review] \hfill \\
Compliance and security can be automated and violations can be handled according to a customizable resolvement procedure.  

\end{description}

\subsection{Configuration Management Tools}

\todo{Needs some work}
%why recent interest for config management tools (DevOps)
%trends
%why are we looking at chef & puppet


There are many commercial configuration management tools available today as well as open source software. The three most commonly used configuration tools are CFEngine, Chef and Puppet \cite{pandey2012investigating}. CFEngine was created in 1993 as a research project and gradually transformed into a mature open source solution for configuration management\cite{Zamboni:2012:LCA:2341102}. Puppet and Chef are newer developments and both maintain active open source communities \cite{pandey2012investigating}. This is also why these two tools are subject of the report.
