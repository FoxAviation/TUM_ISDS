\subsection{Chef}
	Found on \url{https://docs.chef.io/chef_overview.html}
	\begin{itemize}
		\item Components
			\begin{itemize}
				\item Chef DK (Chef Development Kit)
					\begin{itemize}
						\item Computers running Chef DK are called Workstations
						\item Creation of cookbooks
						\item Test of cookbooks with Test Kitchen
							\begin{itemize}
								\item Describe Test Kitchen here
							\end{itemize}
						\item Components of workstations
							\begin{itemize}
								\item Knife
									\begin{itemize}
										\item Interface between local chef-repo and Chef server
									\end{itemize}
								\item The chef-repo
									\begin{itemize}
										\item Cookbook storage
										\item "The chef-repo should be synchronized with a version control system (such as git), and then managed as if it were source code" \url{https://docs.chef.io/workstation.html#configure-ruby-environment}
									\end{itemize}
								\item knife.rb
									\begin{itemize}
										\item File to specify configuration details for knife
									\end{itemize}
							\end{itemize}
					\end{itemize}
				\item Chef Server
					\begin{itemize}
						\item Hub for configuration data (cookbooks)
						\item Pull configuration: Nodes pull cookbooks from server
					\end{itemize}
				\item Node
					\begin{itemize}
						\item Client software must be installed on each node
						
					\end{itemize}
				\item Chef Supermarket
				\begin{itemize}
					\item Sharing and management of community cookbooks
				\end{itemize}
			\end{itemize}
		\item Cookbooks contain
			\begin{itemize}
				\item attributes
				\item cookbook\_file
				\item libraries: Ruby code can be included in a cookbook
				\item metadata: Stored in \textit{metadata.rb}. Helps the server deploy the cookbooks to the nodes correctly
				\item recipes
					\begin{itemize}
						\item Authored in Ruby
						\item Collection of ressources
						\item Must define everything that is needed to configure the node
					\end{itemize}
				\item ressources
					\begin{itemize}
						\item Describes the desired state for a configuration item
						\item Describes the steps to achieve the desired state
						\item Contains ressource type
						\item Grouped into recipes
					\end{itemize}
				\item templates
					\begin{itemize}
						\item Used to dynamically generate static text files
						\item May contain Ruby
						\item Intended to manage configuration files
					\end{itemize}
				\item tests
			\end{itemize}
	\end{itemize}

\subsection{Puppet}
	\begin{itemize}
		\item Uses a master-slave architecture
		\item Uses pull config
		\item Ressource management:
		\begin{itemize}
			\item Manifests describe the node configuration
			\item Groups of ressources can be organized into classes $\Rightarrow$ i.e. config for entire application can be grouped
			\item Modules combine manifests and data to improve code organization
		\end{itemize}
		\item Server node connection via SSL works as follows:
		\begin{enumerate}
			\item Node sends normalized data to the Puppet master
			\item Server uses this data to compile a catalog, that specifies how the node should be configured
			\item The node reports back the successful config to the master (Visible on the Puppet Dashboard)
		\end{enumerate}
	\end{itemize}

\subsection{Evaluation}

Actual hard work happens here - many thoughts!


\section{Conclusions}

Brilliant results!

