\subsection{Chef}

	\begin{itemize}
		\item Components
		\begin{itemize}
			\item Chef DK
				\begin{itemize}
					\item "Location where users interact with chef"
					\item Creation of cookbooks
					\item Test of cookbooks with Test Kitchen
						\begin{itemize}
							\item Describe Test Kitchen here
						\end{itemize}
				\end{itemize}
			\item Chef Server
				\begin{itemize}
					\item Hub for configuration data (cookbooks)
					\item Pull configuration: Nodes pull cookbooks from server
				\end{itemize}
			\item Node
				\begin{itemize}
					\item Client software must be installed on each node
					
				\end{itemize}
		\end{itemize}
	\end{itemize}

\subsection{Puppet}
	\begin{itemize}
		\item Uses a master-slave architecture
		\item Uses pull config
		\item Ressource management:
		\begin{itemize}
			\item Manifests describe the node configuration
			\item Groups of ressources can be organized into classes $\Rightarrow$ i.e. config for entire application can be grouped
			\item Modules combine manifests and data to improve code organization
		\end{itemize}
		\item Server node connection via SSL works as follows:
		\begin{enumerate}
			\item Node sends normalized data to the Puppet master
			\item Server uses this data to compile a catalog, that specifies how the node should be configured
			\item The node reports back the successful config to the master (Visible on the Puppet Dashboard)
		\end{enumerate}
	\end{itemize}

\subsection{Evaluation}

Actual hard work happens here - many thoughts!


\section{Conclusions}

Brilliant results!

