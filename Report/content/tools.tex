% WRITTEN BY: Felix
\section{Configuration Management Tools}

%why recent interest for config management tools (DevOps)
%trends
%why are we looking at chef & puppet


There are many commercial configuration management tools available today as well as open source software. The three most commonly used configuration tools are CFEngine, Chef and Puppet \cite{pandey2012investigating}. CFEngine was created in 1993 as a research project and gradually transformed into a mature open source solution for configuration management\cite{Zamboni:2012:LCA:2341102}. Puppet and Chef are newer developments and both maintain active open source communities \cite{pandey2012investigating}. This is also why these two tools are subject of the report.


\subsection{Puppet}

In 2003, Luke Kanies announced an alternative to CFEngine 2 which was released in 2001 \cite{pandey2012investigating}. Written by a system administrator, Puppet provided an OSAL \cite{kanies2006puppet} addressing the needs of infrastructure operators.

Puppet was initially released under the GNU General Public License (GPL) and re-licensed in 2011 under the more permissive Apache License 2.0 in order to drive adoption and partnerships with companies \cite{puppetcomapache}. There are two versions of Puppet available. Open Source Puppet comes with a limited set of features and no commercial support, but is available free of charge. The paid Enterprise package aims for handling more complex infrastructure and provides more functionality such as reporting tools, advanced orchestration capabilties and web-based user interfaces\cite{puppetcomenterprise}.


%\begin{itemize}
%	\item Short introduction to history (maybe?)
%	\item Idea: Balance between detail-hiding and required customization
%\end{itemize}
%
%\begin{itemize}
%	\item Consists of Puppet server and Puppet agents
%	\item Server (\url{https://puppet.com/docs/puppet/4.9/architecture.html})
%	\begin{itemize}
%		\item One or more servers can run the Puppet master application
%		\item 
%	\end{itemize}
%	\item Uses a master-slave architecture
%	\item Uses pull config
%	\item Ressource management:
%	\begin{itemize}
%		\item Manifests describe the node configuration
%		\item Groups of ressources can be organized into classes $\Rightarrow$ i.e. config for entire application can be grouped
%		\item Modules combine manifests and data to improve code organization
%	\end{itemize}
%	\item Server node connection via SSL works as follows:
%	\begin{enumerate}
%		\item Node sends normalized data, called facts, to the Puppet master and requests a catalog
%		\item Communication between Server and Client via HTTPS and client-verification (SSL certificate)
%		\item Server uses this data to compile a catalog, that specifies how the node should be configured
%		\item After applying the catalog, the agent submits a report to the Puppet master: The node reports back the successful config to the master (Visible on the Puppet Dashboard)
%	\end{enumerate}
%	\item Puppet can run as stand-alone architecture
%	\item Puppet language
%	\begin{itemize}
%		\item \textbf{Declarative} language
%		\item \textbf{Resource} can describe a single file or packet
%		\item Groups of resources can be organized as \textbf{Classes}
%		\item Files
%		\begin{itemize}
%			\item Files are called \textbf{Manifests}
%			\item Are named with \inlinedef{.pp}
%			\item Must use UTF-8
%			\item File example\\
%			\begin{minted}[obeytabs,gobble=4, linenos, bgcolor=white, fontsize=\tiny]{puppet}
%			case $operatingsystem {
%			centos, redhat: { $service_name = 'ntpd' }
%			debian, ubuntu: { $service_name = 'ntp' }
%			}
%			
%			package { 'ntp':
%			ensure => installed,
%			}
%			
%			service { 'ntp':
%			name      => $service_name,
%			ensure    => running,
%			enable    => true,
%			subscribe => File['ntp.conf'],
%			}
%			
%			file { 'ntp.conf':
%			path    => '/etc/ntp.conf',
%			ensure  => file,
%			require => Package['ntp'],
%			source  => "puppet:///modules/ntp/ntp.conf",
%			# This source file would be located on the Puppet master at
%			# /etc/puppetlabs/code/modules/ntp/files/ntp.conf
%			}
%			\end{minted}
%			\item Advertised benefits
%			\begin{itemize}
%				\item Easy to read "Self-documenting"
%				\item consistent nodes
%				\item well-tested by large community
%			\end{itemize}
%		\end{itemize}
%	\end{itemize}
%\end{itemize}


\subsection{Chef}

In January 2009, Jesse Robbins introduced a new systems integration framework called Chef \cite{chefcomannouncement} \cite{mittechreviewrobbins}. With the concept of providing a simple and flexible interface to define infrastructure as code, Chef quickly gained popularity \cite{pandey2012investigating}.

Similar to Puppet, Chef is available as free open source as well as a paid version called Chef Automate. Similiar to Puppet, the open source version is limited in regard to features. In addition to the open source capabilities, Chef Automate provides additional tools for  infrastructure monitoring and error diagnostics. Commercial support is included as well \cite{chefiodatasheet}.

%Short intro to chef and key design principles
%Found on \url{https://docs.chef.io/chef_overview.html}
%\begin{itemize}
%	\item Components
%	\begin{itemize}
%		\item Chef DK (Chef Development Kit)
%		\begin{itemize}
%			\item Computers running Chef DK are called Workstations
%			\item Creation of cookbooks
%			\item Test of cookbooks with Test Kitchen
%			\begin{itemize}
%				\item Describe Test Kitchen here
%			\end{itemize}
%			\item Components of workstations
%			\begin{itemize}
%				\item Knife
%				\begin{itemize}
%					\item Interface between local chef-repo and Chef server
%				\end{itemize}
%				\item The chef-repo
%				\begin{itemize}
%					\item Cookbook storage
%					\item "The chef-repo should be synchronized with a version control system (such as git), and then managed as if it were source code"\\ {\tiny \url{https://docs.chef.io/workstation.html#configure-ruby-environment}}
%				\end{itemize}
%				\item knife.rb
%				\begin{itemize}
%					\item File to specify configuration details for knife
%				\end{itemize}
%			\end{itemize}
%		\end{itemize}
%		\item Chef Server
%		\begin{itemize}
%			\item Hub for configuration data (cookbooks)
%			\item Pull configuration: Nodes pull cookbooks from server
%			\item Features
%			\begin{itemize}
%				\item Search any type of data that is indexed by the Chef server (features wildcards, etc.)
%				\item Management of
%				\begin{itemize}
%					\item Nodes
%					\item Cookbooks and recipes
%					\item Roles
%					\item Stores of JSON data (data bags), including encrypted data
%					\item Environments
%					\item User accounts and user data
%				\end{itemize}
%				\item data bag
%				\begin{itemize}
%					\item Global variable that is stored as JSON data
%					\item Accessible from Chef server
%					\item Indexed for searching
%				\end{itemize}
%				\item Policy
%				\begin{itemize}
%					\item Role
%					\begin{itemize}
%						\item Defines patterns and processes that exists across nodes
%						\item Chef client merges attributes and run-lists with assigned roles
%					\end{itemize}
%				\end{itemize}
%				\item Environment
%				\begin{itemize}
%					\item Maps the real-life workflow to the configuration items of Chef
%					\item Can be associated with one or more cookbook versions
%				\end{itemize}
%				\item Run-list
%				\begin{itemize}
%					\item Ordered list of roles and/or recipes
%					\item Items run in the order defined in the run-list
%					\item Can be node-specific
%					\item Stored as part of the node object on the Chef server
%					\item Maintenance with knife or Chef Automate
%				\end{itemize}
%			\end{itemize}
%		\end{itemize}
%		\item Chef client
%		\begin{itemize}
%			\item Must be installed on each node
%			\item Performs
%			\begin{itemize}
%				\item Registration and authentication of the node with the chef server
%				\item Building the node object
%				\item Synchronization of cookbooks
%				\item Compilation of the resource collection
%				\item Configuration of the node
%				\item Exception and notification handling
%			\end{itemize}
%		\end{itemize}
%		\item Ohai
%		\begin{itemize}
%			\item Collects system configuration data for use within cookbooks
%			\item Includes many built-in plugins to detect state
%			\item Attributes contain: OS, Network, Memory, Disk, CPU, Kernel, host names, virtualization, etc.
%		\end{itemize}
%		\item Chef Supermarket
%		\begin{itemize}
%			\item Sharing and management of community cookbooks
%		\end{itemize}
%	\end{itemize}
%	\item Cookbooks contain
%	\begin{itemize}
%		\item attributes
%		\item cookbook\_file
%		\item libraries: Ruby code can be included in a cookbook
%		\item metadata: Stored in \textit{metadata.rb}. Helps the server deploy the cookbooks to the nodes correctly
%		\item recipes
%		\begin{itemize}
%			\item Authored in Ruby
%			\item Collection of ressources
%			\item Must define everything that is needed to configure the node
%		\end{itemize}
%		\item ressources
%		\begin{itemize}
%			\item Describes the desired state for a configuration item
%			\item Describes the steps to achieve the desired state
%			\item Contains ressource type
%			\item Grouped into recipes
%		\end{itemize}
%		\item templates
%		\begin{itemize}
%			\item Used to dynamically generate static text files
%			\item May contain Ruby
%			\item Intended to manage configuration files
%		\end{itemize}
%		\item tests
%	\end{itemize}
%\end{itemize}

