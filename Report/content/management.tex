%WRITTEN BY: Daniel
\subsection{Management}

The configuration management tool does not operate in a vacuum but needs to work well together with existing tools, infrastructure and software development procedures.

\subsubsection{Testing}\hfill\\
Similiar to application code, configuration changes need to be tested before they are applied to the production system. There are a number of different tests that can be used to verify configurations.

\begin{description}
	
	\item[Syntax Checking] \hfill \\ 
	Identification of syntactical mistakes by simulating the interpreter. This is often provided as part of the editing experience inside an editor.
	
	\item[Unit Tests] \hfill \\
	Testing of single functions and units of the configuration to ensure they are behaving as expected.
	
	\item[System / Integration Tests]  \hfill \\
	Testing of multiple configuration components and functions in a simulated environment reflecting the live system.
	
	\item[Dry Runs] \hfill \\
	Configuration Management tools provide facilities to simulate the application of a configuration but does not actually apply it. The simulation is often referred to as a \textit{Dry Run}.	
	
\end{description}

Puppets CLI offers a syntax parser to check configurations and there exists a tool called \texttt{puppet-lint} which checks code against the published Puppet Styleguide to catch common errors and bad practices. To run Integration, System and Acceptance Tests there are a number of tools such as Cucumber or Beaker \cite{verifypuppet}. Puppet also has a dry-run feature implemented.

Since Chefs input specification is a Ruby-DSL, the existing tools for syntax checking, linting and testing that exist in the Ruby ecosystem can be used to verify Chef configurations which is a great benefit. For advanced integration tests there is \textit{Kitchen} which is a platform for running infrastructure tests \cite{kitchen}. It is CM tool agnostic so it can also be used for testing Puppet code. While Chef supports a dry-run mode called \texttt{why-run} it is actually not recommended to use since it can only forecast the expected result of configuration changes but its output might be misleading. According to a Chef product manager this feature was integrated early in Chefs development to reach feature parity with Puppet and Cfengine but should not be used anymore. Instead it is recommended to put more emphasis on auditing and monitoring throughout a configuration change\cite{whyrun}.

Both ecosystems provide tools and good practices to verify and test configuration in isolation as well as inside a test environment. Chef benefits from existing frameworks inside the Ruby ecosystem while the community around Puppet has build specialized tools to enable configuration testing.


\subsubsection{Monitoring \& Integrations}\hfill\\

Essential for configuration management tools is the integration into the existing tool landscape. Integrations into 

\begin{itemize}
\item Version Control Systems
\item Continous Integration Servers
\item Cloud Platforms
\item Monitoring Systems
\item Access Control Systems
\end{itemize}

are either offered as part of the commercial offering of both tools respectively or contributed by the community.

