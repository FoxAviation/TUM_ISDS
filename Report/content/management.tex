%WRITTEN BY: Daniel
\subsection{Management}

The configuration management tool does not operate in a vacuum but needs to work well together with existing tools, infrastructure and software development procedures.

\subsubsection{Testing}\hfill\\
Similiar to application code, configuration changes need to be tested before they are applied to the production system. There are a number of different tests that can be used to verify configurations.

\begin{description}
	
	\item[Syntax Checking] \hfill \\ 
	Identification of syntactical mistakes by simulating the interpreter. This is often provided as part of the editing experience inside an editor.
	
	\item[Unit Tests] \hfill \\
	Testing of single functions and units of the configuration to ensure they are behaving as expected.
	
	\item[System / Integration Tests]  \hfill \\
	Testing of multiple configuration components and functions in a simulated environment reflecting the live system.
	
	\item[Dry Runs] \hfill \\
	Configuration Management tools provide facilities to simulate the application of a configuration but does not actually apply it. The simulation is often referred to as a \textit{Dry Run}.	
	
\end{description}


\subsubsection{Monitoring \& Integrations}\hfill\\

Essential for configuration management tools is the integration into the existing tool landscape. Integrations into 

\begin{itemize}
\item Version Control Systems
\item Continous Integration Servers
\item Cloud Platforms
\item Monitoring Systems
\item Access Control Systems
\end{itemize}

are either offered as part of the commercial offering of both tools respectively or contributed by the community.

