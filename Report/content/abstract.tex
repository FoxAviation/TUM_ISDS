The deployment of internet-scale distributed systems requires a structured process for installing, configuring and running components on multiple hosts to form a unified system. With increasing system complexity the traditional approach of manually installing the software and associated dependencies is not feasible anymore. Over the years, a number tools have been developed to automatically configure fleets of servers - Configuration Management Tools. In order to be able to make a sensible decision of what tool to use in a certain context one has to inspect what makes a configuration management tool what problems it solves. The report introduces the discipline of software configuration management and the capabilities of associated tools by introducing two of the most popular configuration management tools, Chef and Puppet. We conclude that while technical factors play into the decision for a configuration management tool, non-technical factors such as maturity, support and community activity should be considered as well.