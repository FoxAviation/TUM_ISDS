%WRITTEN BY: Felix
\section{Characteristics of Configuration Management Tools}

\subsection{Specification}

\subsubsection{Input specification}

As specification paradigms of configuration management tools, there are two approaches do specify infrastructure as code. 

Firstly, declarative languages allow the description of the desired state of a system. Popular declarative languages are i.e. HTML or SQL, where you only describe the desired output of your query, but barely care about the actual execution of the required operations itself. In CM, the desired system state is compared to the actual state of each individual node. The translation agent then derives commands to configure each node appropriately \cite{delaet2010survey}. Secondly, i.e. popular languages like C and Java are called imperative. These describe the process to achieve a desired result, given a specified input. In terms of configuration management, the use of an imperative language requires to describe all steps to configure your systems in the desired way.

Reductive labs have specified their own language for Puppet, named "Puppet Language". This declarative language is inspired by the Nagios file format to be easily accessible by system administrators without requiring much programming experience. Puppet advertises this language  as self-documenting.
% TODO: Describe ressources in the puppet language
Similar to the main method in languages like C, Puppet begins the compilation process with the "main manifest", where a manifest is the term puppet uses for naming configuration files \cite{puppetcomlangsum}. These manifests are compiled into catalogs, which contain the data in the xml format.

Similar to Puppet's manifest files, Chef uses Recipes to specify configurations, which is done using their own Domain Specific Language as well as Ruby \cite{pandey2012investigating}. This language allows both, declarative and imperative elements and is designed to be especially human-understandable. With Ruby, Chef also stresses the so called "There is more than one way to do it" (TMTOWTDI) \todo{Rewrite this sentence} mentality, meaning that there are usually multiple possibilities to achieve a similar result for a specific configuration task. This enhances the user's flexibility to configure infrastructure \cite{https://docs.chef.io/recipes.html}. Recipes must be stored in a cookbook, a class-like structure, that can, but is not limited to Attributes and Recipes. A cookbook contains all necessary resources and policies for a configuration item \cite{chefiocookbooks}. 

Apart from the programming language, the interface of the user with the CM tool plays an important role in terms of usability. A graphical user interface typically provides a more intuitive way of interacting with the software. For this reason, GUI-based approaches allow to get in touch with the tool quicker. However, command-line interface allow for a steeper learning curves and the ability to not only interact with the tool faster, but also enable easy access through scripts \cite{delaet2010survey}.

\subsubsection{Abstraction mechanisms}

Configuration Management can be used for managing a wide variety of hardware. From servers and desktop computers to network devices, each device may look different from a configuration point of view. To manage these devices nevertheless, Puppet and Chef each provide abstraction mechanisms. These help to provide a uniform interface, even for complex heterogeneous infrastructure. Abstraction policies can range from high-level specifications, such as the specification of required database servers, down to the configuration of a file at bit level \cite{delaet2010survey}.

The fundamental concept of abstraction in configuration management is resources, which describe the desired state for a configuration item amongst other things \cite{chefioresource}. These can be files and directories, as well as a user account or a service (i.e. SSH). Chef and Puppet each bring their own set of definable resources and with it their own terminology.

Puppet defines a resource with a type, a name and a set of attributes \cite{pandey2012investigating}. If a tar.gz package would be required to be installed at version 1.16.1, the definition could look as follows:

\todo{Find a nicer example}
\begin{minted}[obeytabs,gobble=0, linenos, bgcolor=white, fontsize=\normalsize]{puppet}
package { 'tar':
	ensure => "1.16.1",
}
\end{minted}

In contrast, a Chef resource is describes by a Ruby block that contains four components: a type, a name and the properties as well as actions required to perform the desired operation. The following example  shows the Ruby DSL code to again install a tar.gz package of version 1.16.1 \cite{chefioresource}:

\todo{Find a nicer example}
\begin{minted}[obeytabs,gobble=0, linenos, bgcolor=white, fontsize=\normalsize]{ruby}
package 'tar' do
	version '1.16.1'
	action :install
end
\end{minted}

\subsubsection{Modularization}

Modern IT infrastructure comprises of countless applications and services. Configuring every device independently would imply a low of repetitive work and be error prone. Similar to the concept of inheritance, that exists in object-oriented programming, modularization enables to abstract from specific devices and create groups and associations with other nodes, as well as common configuration parameters. These groups can not only be flat, but may have engraved some type of hierarchical structure \cite{delaet2010survey}. This allows i.e. for an abstract specification of the infrastructure and specific configuration for devices, that inherit their parameters from the abstract description. Such a hierarchical system remains very flexible, because configurations mostly take place on abstract layers and low-level components can easily be scaled.

Within an organization of even entire communities, configuration tasks can be very similar. This could for example be setting up an SSH service or the configuration of a DNS system. To tackle the issue of rewriting configurations that already exist or can be easily recycled, Puppet and Chef have developed their own pool, where users can publish or search for and reuse existing configurations with the entire community or locally within their organization.

\todo{Describe forge and supermarket}

\subsubsection{Dependency modeling}

In complex IT-infrastructures, configuration tasks come with some implicit dependencies, that must be or remain fulfilled. This may for example be the configuration of some access rights, which require the successful installation of the affected service on the device. Explicitly handling such relations help keeping the infrastructure consistent.

In configuration management, one distinguishes thee types of relations \cite{delaet2010survey}:
\begin{description}
\item[Instance-Instance] \hfill \\ 
Instance-to-instance relations comprise multiple instances of configuration items on either the same or different devices. A web-server might be dependent on the existence of a database.
\item[Instance-Parameter] \hfill \\ 
Instances can not only depend on other instances, but also external parameters. There may for example be a server, that requires a specific port to be forwarded from the router by setting an appropriate entry in a configuration file.
\item[Parameter-Parameter] \hfill \\ 
Finally, parameters can relate to other parameters. There may be a setup including some servers and a router, where the router forwards connections to the servers IP. So whenever the local IP address of the server changes, the router must set its forwarding record appropriately.
\end{description}

Puppet has realized modeling relations though so called metaparameters, which affect how Puppet executes the configuration of nodes. There is a wide variety of such metaparameters specified, which affect different characteristics of the configuration. This contains a notification system, that refreshes resources that have subscribed to changes of a specific resource. Scheduling enables i.e. the execution of a command in frequent time intervals.

\begin{minted}[obeytabs,gobble=0, linenos, bgcolor=white, fontsize=\normalsize]{puppet}
schedule { 'everyday':
	period => daily,
	range  => "2-4"
}
exec { "/usr/bin/apt-get update":
	schedule => 'everyday'
}
\end{minted}

The capabilities of the metaparameter system do not end here, but also contains other powerful commands, like specifying the level of detail in which logs are created and much more  \cite{puppetcommetaparameter}.

Chef provides similar capability though a set of tools. The notification system is, just like in Puppet, based on notifies and subscriptions, but not limited to that. Timers allow for explicit definition of execution order and delays in between commands. It can also be specified, that a configuration task should be executed immediately after a prior operation. This allows i.e. for restarting a service immediately prior or after the update of a resource. Finally, jobs can be defined so that they run before the actual execution phase of a chef run, where the configuration takes place under normal circumstances \cite{chefiocommonfunc}.